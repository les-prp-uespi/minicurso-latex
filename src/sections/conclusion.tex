\section{Conslusão}

\begin{frame}
\frametitle{Nota sobre o material} 

\begin{itemize}
  \item Este material foi criado com base em duas referencias principais:
  \begin{itemize}
		\item Curso de extensão em \LaTeX mistrado por \textit{Messias Alves} em 2008.
    {\color{darkcerulean} /*Parte sobre a criação de documentos*/}
		\item Tutorial de Beamer em Beamer, do \textit{Prof. Charles T. Batts} de
		2007. {\color{darkcerulean} /*Parte sobre a criação de apresentações*/}
  \end{itemize}
\end{itemize}
\end{frame}


\begin{frame}
\frametitle{Links Úteis} 

\begin{itemize}
  \item \url{http://latex.simon04.net/}
  \item \url{http://deic.uab.es/~iblanes/beamer_gallery/index_by_theme.html}
  \item \url{http://texdoc.net/texmf-dist/doc/latex/beamer/doc/beameruserguide.pdf}
  \item \url{http://www.stdout.org/~winston/latex/latexsheet.pdf}
  \item \url{http://en.wikibooks.org/wiki/LaTeX}
  \item {\Large \color{darkcerulean} \url{http://tex.stackexchange.com/}}
\end{itemize}
\end{frame}


\begin{frame}
\frametitle{Obrigado}

Happy \LaTeX\ coding!

Obrigado por ter tirado um tempo para estar aqui e acompanhar este tutorial de \LaTeX\. Agora você
deve ter um conhecimento básico para começar a criar seus documentos e apresentações com alta
qualidade.
\begin{figure}
\includegraphics[scale=.05]{../img/thank-you}
\end{figure}

\end{frame}

