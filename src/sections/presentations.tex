\section{Apresentações com Beamer}


\subsection[Estrutura]{Estrutura}
% =============== %
%     Frame       %
\begin{frame}
\frametitle{Sobre o Beamer}
\begin{itemize}
  \item Os comandos padrões e \LaTeXe\ também funcionam no Beamer
  \item Súmários podem ser gerados automáticamente
  \item Você pode facilmente criar efeitos dinâmicos
  \item A aparência pode ser mudada com uso de temas à seu gosto
  \item Os temas disponíveis por padrão são bem estruturados e fáceis de ler. O que torna a
  apresentação mais profissional e fácil da audiência seguir.
\end{itemize}
\end{frame}

% =============== %
%     Frame       %
\begin{frame}
\frametitle{Sobre o Beamer}
\begin{itemize}
  \item A aparência, cores e fontes utilizada na apresentação podem ser facilmente alterada de forma
  \textit{global}, mas alterações podem ser feitas de forma \textit{local}
  \item Você pode cirar apresentações usando o mesmo código utilizado no seu artigo \LaTeX
  \item A saída produzida é típicamente um \textcolor{darkcerulean}{\textit{.pdf} file}, o que
  facilita a apresentação em qualquer plataforma
  \item {\color{darkcerulean}Sua apresntação irá ter a mesma estrutua, independente de qual
  computador ou visualizador está sendo utilizado}
\end{itemize}
\end{frame}

\begin{frame}
\frametitle{Onde achar o Beamer?}
\begin{center}
Beamer está disponível para download \textcolor{darkcerulean}{\textit{gratuitamente}} em:\newline
\textcolor{darkcerulean}{\url{https://bitbucket.org/rivanvx/beamer/wiki/Home}}
\end{center}
\begin{center}
Existe bastante coisa sobre Beamer na Internet e existe também uma 
\textit{documentação} Beamer disponível no repositório acima e no endereço abaixo: \newline
\url{http://www.ctan.org/tex-archive/macros/latex/contrib/beamer/doc/}
\end{center}
\end{frame}


% =============== %
%     Frame       %
\begin{frame}
\frametitle{Usando templates prontos}
\begin{itemize}
  \item A maneira mais rápida de iniciar a desenvolver apresentações com Beamer é utilizar-se de
  templates prontos.
  \item Vários templates prontos estão disponíveis no repositório do Beamer
  \item Um exemplo pode ser encontrado seguindo este caminho:
  \url{beamer/solutions/conference-talks/conference-ornate-20min.en.tex}
  \item Copie o arquivo e modifique os conteúdos.
\end{itemize}
\end{frame}

% =============== %
%     Frame       %
\begin{frame}
\frametitle{Para testar suas apresentações}
\begin{itemize}
  \item Para ver como é uma apresentação, compile o código \LaTeX\ \textcolor{darkcerulean}{duas}
  vezes
  \item Abra o arquivo \textit{.pdf} com o visualizador disponível e utilize em modo ``Tela Cheia''
  \item O sumário gerado tem \textit{hyperlinks} nas seções e subseções, além de uma linha
  auxiliar com botões de navegação
\end{itemize}
\end{frame}

% =============== %
%     Frame       %
\begin{frame}[fragile]
\frametitle{Frames}
\scriptsize

\begin{itemize}
  \item  Cada projeto Beamer é feito de uma série de \textcolor{darkcerulean}{\textit{frames}}. Cada frame
produz um ou mais slides, dependendo da existência ou não de ``\textit{overlays}'', as quais serão
discutidas mais tarde.
  \item  A opção \verb|[plain]| causa a supressão de ``cabeçalho'', ``rodapé'', e ``barra lateral''.
  Útil pra exibir figuras grandes.

\end{itemize}
\begin{block}{Um frame básico}
	\verb|\begin{frame}[<alignment>]| \newline
  	\verb|\frametitle{Frame Title Goes Here}|\newline
	\verb|	Texto do frame e/ou o código LaTeX.|\newline
	\verb|\end{frame}|
\end{block} 
\end{frame}

% =============== %
%     Frame       %
\begin{frame}[fragile]
\frametitle{Frames}
\scriptsize
\begin{itemize}
  \item Para compor frames basta escrever seu texto ou código \LaTeX\ entre os
comandos \verb|\begin{}| e \verb|\end{}| frame.
  \item Os frames são centralizados \verb|[c]| por padrão. Os valores \verb|[t]| (alinhamento
superior) e \verb|[b]|  (alinhamento inferior) também são aceitos.
\end{itemize}

\begin{block}{Um frame básico}
	\verb|\begin{frame}[t]|\newline
  	\verb|\frametitle{Frame Title Goes Here}|\newline
	\verb|	Texto do frame e/ou o código LaTeX.|\newline
	\verb|\end{frame}|
\end{block}
\end{frame}

% =============== %
%     Frame       %
\begin{frame}[fragile]
\frametitle{``Capa'' para a apresentação }
O frame de capa mostra somente as informações inserida no início do documento:

\begin{block}{Um frame básico}
	\verb|\begin{frame}|\newline
  	\verb|   \titlepage|\newline
	\verb|\end{frame}|
\end{block}

\end{frame}


% =============== %
%     Frame       %
\begin{frame}[fragile]
\frametitle{``Capa'' para a apresentação }
Por padrão, o comando \verb|\titlepage| cria uma página que inclui:
\begin{itemize}
  \item Título
  \item Autor
  \item Afiliação
  \item Data
  \item Imagem (logo)
\end{itemize}

Caso algum desses valores não seja declarados no preâmbulo, eles não seram incluídos do slide de
capa.
\end{frame}


% =============== %
%     Frame       %
\begin{frame}[fragile]
\frametitle{Slide de Sumário}
O comando \verb|\tableofcontents| cria dinamicamente o sumário baseado na estrutura que você
definiu
 
\begin{block}{Slide de Sumário}
\scriptsize
	\verb|\begin{frame}|\newline
	\verb|  \frametitle{Sumário}|\newline
  	\verb|  \tableofcontents[ pausesections]|\newline
	\verb|\end{frame}|
\end{block}

Perceba que o argumento \texttt{pausesections} permite que os items apareçam seção à seção.
\end{frame}

% =============== %
%     Frame       %
\begin{frame}[fragile]
\frametitle{Juntando as coisas}
 
\begin{block}{Exemplo}
\scriptsize
	\verb|\begin{frame}|\newline
	\verb|  \titlepage|\newline
	\verb|\end{frame}|
	\newline
	\verb|\begin{frame}|\newline
	\verb|  \frametitle{Sumário}|\newline
  	\verb|  \tableofcontents[ pausesections]|\newline
	\verb|\end{frame}|\newline
	\newline
 	\verb|\begin{frame}|\newline
	\verb|  \frametitle{Introdução}|\newline
  	\verb|  Corpo do texto ou código LaTeX.|\newline
	\verb|\end{frame}|
\end{block}

\end{frame}


% =============== %
%     Frame       %
\begin{frame}[fragile]
\frametitle{Overlays}
\begin{itemize}
  \item \textit{Overlays} permitem que seus slides apareçam incrementalmente.
  \item Mais especificamente, em Beamer, \textcolor{darkcerulean}{overlays} controlam a ordem na
  qual as partes do frame aparecem.
  \item Uma maneira fácil de implementar overlays é usar o comando \verb|\pause| entre as partes que
  devem aparecer serparadamente
\end{itemize}
\end{frame}


% =============== %
%     Frame       %
\begin{frame}[fragile]
\frametitle{Overlays}
Por exemplo: 

\begin{verbatim}
\textbf{Step1:} Compute the maximal suffix of $w$
with respect to $\preceq_l$ (say $v$) and the
maximal suffix of $w$ with respect to $\preceq_r$
(say $v’$).
\pause

\textbf{Step 2:} Find words $u$, $u’$ such that
$w = uv = u’v’$.
\pause

\textbf{Step 3:} If $|v| \le |v’|$, then output
$(u,v)$. Otherwise, output$(u’,v’)$.
\end{verbatim}
\end{frame}

% =============== %
%     Frame       %
\begin{frame}[fragile]
\frametitle{Overlays (Resultado)}
\textbf{Step1:} Compute the maximal suffix of $w$
with respect to $\preceq_l$ (say $v$) and the
maximal suffix of $w$ with respect to $\preceq_r$
(say $v’$).
\pause

\textbf{Step 2:} Find words $u$, $u’$ such that
$w = uv = u’v’$.
\pause

\textbf{Step 3:} If $|v| \le |v’|$, then output
$(u,v)$. Otherwise, output$(u’,v’)$.
\end{frame}

% =============== %
%     Frame       %
\begin{frame}[fragile]
\frametitle{Especificação de Overlays}
São feitas com os símbolos ($<$, $>$) e indicam quais partes devem aparecer

A especificação \verb|<1->| diz ``mostre do slide 1 em diante.'' \verb|<1-3>| diz ``mostre do
slide 1 ao 3.'' \verb|<-3,5-6,8->| diz ``mostre todos os slides, exceto os slides 4 e 7.''

Um exemplo:
\begin{columns}
	\column{.5\textwidth}
	\small
\begin{verbatim}
	\begin{itemize}
   \item<1>    $abcadcabca$
   \item<1-2>  $abcabcabca$
   \item<1-2>  $accaccacca$
   \item<1>    $bacabacaba$
   \item<1,3>  $cacdaccacc$
   \item<1-2>  $caccaccacc$
\end{itemize}
\end{verbatim}
	\column{.5\textwidth}
	\small
\begin{itemize}
   \item<1>    $abcadcabca$
   \item<1-2>  $abcabcabca$
   \item<1-2>  $accaccacca$
   \item<1>    $bacabacaba$
   \item<1,3>  $cacdaccacc$
   \item<1-2>  $caccaccacc$
\end{itemize}\end{columns}

\end{frame}



% =============== %
%     Frame       %
\begin{frame}[fragile]
\frametitle{Especificação de Overlays}
Podem também ser utilizadas para dar efeito em partes do texto. Por exemplo, o código abaixo
aplica o comando \verb|\alert{}| somente nos slides especificados:

\begin{columns}
	\small
	\column{.6\textwidth}
\begin{verbatim}
\alert{Todos slides}
\alert<2>{Slide 2}
\alert<3>{Slide 3}
\alert<1,3>{Slides 1 e 3}
\alert<-2,4>{Slides 1, 2 e 4}
\end{verbatim} 
	\column{.4\textwidth}
	\alert{Todos slides}

\alert<2>{Slide 2}

\alert<3>{Slide 3}

\alert<1,3>{Slides 1 e 3}

\alert<-2,4>{Slides 1, 2 e 4}
\end{columns}
\vspace{.5cm}
\textbf{Nota:} Se quiser que cada item de uma lista apareça em ordem, basta usar a opção
\verb|[<+->]|. Exemplo: \verb|\begin{itemize}[<+->]|

\end{frame}


% =============== %
%     Frame       %
\begin{frame}[fragile]
\frametitle{Overlays em ambientes}

Overlays também podem ser utilizados em ambientes

\begin{columns}
	\column{.5\textwidth}
	\scriptsize
\begin{verbatim}
\begin{theorem}<1->
  Um teorema.
\end{theorem}

\begin{proof}<2->
  Uma prova.
\end{proof}
\end{verbatim}
	\column{.5\textwidth}
	\scriptsize
\begin{theorem}<1->
   Um teorema.
  \end{theorem}
  \begin{proof}<2->
   Uma prova.
  \end{proof}
\end{columns}

\end{frame}


% =============== %
%     Frame       %
\begin{frame}
\frametitle{Estrutura dos Frames}
Beamer provêm muitas formas de estruturar seus slides de forma que ele fiquem bem
organizados e fácil de sua audiência seguir. Como exemplos, temos:

\begin{itemize}
  \item Columns
  \item Blocks
  \item Boxes (Borders)
\end{itemize}
\end{frame}

% =============== %
%     Frame       %
\begin{frame}[fragile]
\frametitle{Estrutura dos Frames: Colunas}

O ambiente pode ser chamado como segue:

\begin{block}{}
\small
\verb|\begin{columns}|

\verb|  \column{.xx\textwidth}|

\verb|   Texto ou código da segunda coluna|

\verb|  \column{.xx\textwidth}|

\verb|   Texto ou código da segunda coluna|

\verb|\end{columns}|
\end{block}

Onde \alert{.xx} é porcentagem do slide. 
\end{frame}


% =============== %
%     Frame       %
\begin{frame}[fragile]
\frametitle{Estruturas dos Slides: Blocos}
Blocos podem ser utilizados para serparar uma porção específica do texto do restante do slide:

{\small
\begin{verbatim}
\begin{block}{Introdução à {\LaTeX}}
``Beamer é uma classe {\LaTeX} para criar
 apresentações\ldots''
\end{block}
\end{verbatim}
}

\begin{block}{Introdução à {\LaTeX}}
``Beamer é uma classe {\LaTeX} para criar apresentações\ldots''
\end{block}

\end{frame}


% =============== %
%     Frame       %
\begin{frame}[fragile]
\frametitle{Estruturas dos Slides: Blocos}
Outros ambientes podem ser utilizados como blocos:


\begin{block}{Introduction to {\LaTeX}}
\begin{tabular}{l|l}
\textbf{Conteúdo} & \textbf{Ambiente correspondente} \\
\hline
Genérico & \texttt{block}\\
Teoremas & \texttt{theorem}\\
Lemas & \texttt{lemma}\\
Provas & \texttt{proof}\\
Corolários & \texttt{corollary}\\
Exemplos & \texttt{example}\\
Título em destaque & \texttt{alertblock}\\
\end{tabular}
\end{block}

\end{frame}

% =============== %
%     Frame       %
\begin{frame}[fragile]
\frametitle{Estruturas dos Frames: Colunas e Blocos}

Podemos combinar ``colunas'' e ``blocos'' para fazer uma apresentação mais limpa.

\begin{verbatim}
\begin{columns}[t]
   \column{.5\textwidth}
       \begin{block}{Cabeçalho da Coluna 1}
          Corpo do texto da Coluna 1
       \end{block}
   \column{.5\textwidth}
       \begin{block}{Cabeçalho da Coluna 2}
          Corpo do texto da Coluna 2
       \end{block}
\end{columns}
\end{verbatim}

E temos como resultado\ldots
\end{frame}


% =============== %
%     Frame       %
\begin{frame}[fragile]
\frametitle{Estruturas dos Frames: Colunas e Blocos}


\begin{columns}[t]
   \column{.5\textwidth}
       \begin{block}{Cabeçalho da Coluna 1}
          Corpo do texto da Coluna 1
       \end{block}
   \column{.5\textwidth}
       \begin{block}{Cabeçalho da Coluna 2}
          Corpo do texto da Coluna 2
       \end{block}
\end{columns}
\vspace{1cm}

Perceba que a opção \verb|[t]| adicionado ao ambiente de colunas alinha os blocos por cima para que
eles fiquem na mesma linha vertical, diferentemente de centralizado no slide.
\end{frame}


% =============== %
%     Frame       %
\begin{frame}[fragile]
\frametitle{Estruturas dos Frames: Colunas e Blocos}

Bordas também podem ser utilizadas para adicionar uma organização à sua aprsentação. Com o uso do
pacote \texttt{fancybox} (lembre-se de declarar \verb|\usepackage{facybox}| no preâmbulo). 

\begin{block}{Borda de Textos}
\begin{tabular}{l|l}
\small
\textbf{Comando} & \textbf{Resultado} \\
\hline
\verb|\shadowbox{Texto}| & \shadowbox{Texto}\\
\verb|\fbox{Texto}| & \fbox{Texto}\\
\verb|\doublebox{Texto}| & \doublebox{Texto}\\
\verb|\ovalbox{Texto}| & \ovalbox{Texto}\\
\verb|\Ovalbox{Texto}| & \Ovalbox{Texto}\\
\end{tabular}
\end{block}
\end{frame}


\subsection[Aparência]{Aparência}


% =============== %
%     Frame       %
\begin{frame}[fragile]
\frametitle{Temas}
Temas podem mudar completamente a aparência de sua apresentação. Você escolhe o tema a ser
utilizados usando o comando \verb|\usetheme{}| com um dos seguintes argumentos:

\begin{block}{}
\centering
\begin{tabular}{cccc}
\small
\texttt{Antibes} & \texttt{Boadilla} & \texttt{Frankfurt} & \texttt{Juanlespins} \\
\texttt{Montpellier} & \texttt{Singapore} & \texttt{Bergen} & \texttt{Copenhagen} \\
\texttt{Goettingen} & \texttt{Madrid} & \texttt{Paloalto} & \texttt{Warsaw} \\
\texttt{Berkeley} & \texttt{Darmstadt} & \texttt{Hannover} & \texttt{Malmoe} \\
\texttt{Pittsburgh} & \texttt{Berlin} & \texttt{Dresden} & \texttt{Ilmenau} \\
\texttt{Marburg} & \texttt{Rochester} & & \\
\end{tabular}
\end{block}

\end{frame}


% =============== %
%     Frame       %
\begin{frame}[fragile]
\frametitle{Cores dos Temas}
Se você gosta do ``layout'' de um tema, mas não gosta da cor, você pode facilmente invocar uma nova
cor para o tema substituindo \texttt{default} no comando \verb|\usetheme{default}| inserido no
preâmbulo por um dos seguintes argumentos:

\begin{block}{}
\centering
\begin{tabular}{cccc}
\small
\texttt{albatross} & \texttt{crane} & \texttt{beetle} & \texttt{dove} \\
\texttt{fly} & \texttt{seagull} & \texttt{wolverine} & \texttt{beaver} \\
\end{tabular}
\end{block}

\end{frame}



% =============== %
%     Frame       %
\begin{frame}[fragile]
\frametitle{Cores dos Temas}
Existe também a possibilidade de especificar cores para a parte \textit{interna} ou
\textit{externa} da mesma forma da cor geral do tema: substituindo \texttt{default} no comando
\verb|\usetheme{default}|.


\begin{columns}[t]
   \column{.5\textwidth}
       \begin{block}{Opções parte interna}
          \centering
\begin{tabular}{ccc}
\small
\texttt{lily} & \texttt{orchid} & \texttt{rose} \\
\end{tabular}
       \end{block}
   \column{.5\textwidth}
       \begin{block}{Opções parte externa}
                    \centering
\begin{tabular}{ccc}
\small
\texttt{whale} & \texttt{seahorse} & \texttt{dolphin} \\
\end{tabular}
       \end{block}
\end{columns}

\end{frame}


% =============== %
%     Frame       %
\begin{frame}
\frametitle{Exercício}

% You can create overlays
\begin{enumerate}
  \item Elaborar uma apresentação com as estruturas vistas.
\end{enumerate}
 
\end{frame}

